\begin{Def}[Kuratowski-Mrówa]
  Un espacio topológico $X$ es compacto si y solo si
  para todo espacio topológico $Y$, la segunda proyección
  $\pi_2:X\times Y\longrightarrow Y$ es cerrada.
\end{Def}

Antes de demostrar la equivalencia con la definición usual, se presenta
un lema el cual facilita entender este enunciado y la demostración.

\begin{Lema}
  Sean $X,Y$, espacios topológicos y $f:X\longrightarrow Y$.
  $f$ es cerrada si y solo si, para todo $y\in Y$, $U\subseteq X$
  tal que $f^{-1}(\{y\})\subseteq U$, existe una vecindad $V$ de $y$
  tal que $f^{-1}(V)\subseteq U$.
\end{Lema}
\begin{Demo}
  \begin{enumerate}
    \item Supóngase que $f$ es cerrada y sean $y\in Y$, $U\subseteq X$ tal que $f^{-1}(\{y\})\subseteq U$.
    Entones $X-U$ es cerrado y $f(X-U)$ también. Se define $V=Y-f(X-U)$ que es abierto en $Y$ y además
    $y\in V$, pues de no ser el caso, $y\in f(X-U)\cap f(U)=\varnothing$.

    Se afirma que $f^{-1}(V)\subseteq U$: sea $p\in f^{-1}(V)$, entonces $f(p)\not\in f(X-U)$ y
    $p\not\in X-U$, es decir $p\in U$.
    \item Supóngase que $f$ satisface la hipótesis del recíproco y sean $C\subseteq X$ cerrado, $y\not\in f(C)$.
    Como $C$ es cerrado, entonces $X-U$ es un abierto el cual contiene a $f^{-1}(\{y\})$ y por hipótesis,
    se cuenta con una vecindad $V$ de $y$ tal que $f^{-1}(V)\subseteq X-C$, con lo que
    $V\cap f(C)=\varnothing$ y $y\in\Int(Y-f(C))$. Así, $f(C)$ es cerrado.
  \end{enumerate}
\end{Demo}

\begin{Lema}[Lema del Tubo]
  Sean $X,Y$ espacios topológicos, $y\in Y$ y $W\subseteq X\times Y$ abierto tal que
  $X\times\{y\}\subseteq W$. Si $X$ es compacto entonces existe una vecindad $V$ de $y$ tal
  que $X\times V\subseteq W$.
\end{Lema}
\begin{Demo}
  Dado que $X\times\{y\}$ es homeomorfo a $X$, el cual es compacto, se cuenta con
  una cantidad finita de abiertos básicos $U_1\times E_1,\dots,U_n\times E_n\subseteq W$
  los cuales cubren a $X\times\{y\}$. Sin pérdida de generalidad supóngase que
  cada uno de estos abiertos interseca al conjunto. Se define
  $V=\bigcap_{i=1}^n E_i$ que es abierto en $Y$ y $y\in V$. Sea $(x,z)\in X\times V$.
  Entonces para algún $i\in\{1,\dots,n\}$, $(x,y)\in U_i\times E_i$ y como
  $z\in\bigcap_{i=1}^n E_i$, entonces $(x,z)\in U_i\times E_i$. Así,
  $X\times W\subseteq\bigcup_{i=1}^nU_i\times E_i\subseteq W$.
\end{Demo}

Aplicando el primer lema, el enunciado el teorema presentado resulta equivalente
al siguiente

\begin{Teo}[Kuratowski-Mrówa modificado]
  Un espacio topológico $X$ es compacto (en el sentido de las cubiertas abiertas) si y solo si
  para todo espacio topológico $Y$, se satisface lo siguiente: para todo $y\in Y$,
  $W\subseteq X\times Y$ abierto tal que $X\times\{y\}\subseteq W$, existe una vecindad
  $V$ de $y$ la cual cumple que $X\times V\subseteq U$.
\end{Teo}
\begin{Demo}
  La primera implicación es el lema del tubo.

  Supóngase que X que satisface la propiedad y
  sea $(C_i)_{i\in I}$ una colección de conjuntos 
  cerrados de $X$ la cual satisface la propiedad de las intersecciones 
  finitas (esto es que cualquier intersección finita 
  de los $C_i$ es no vacía). Para demostrar que $X$ es 
  compacto, basta con demostrar que $\bigcap_{i\in I}C_i\neq \varnothing$
  
  Se considera $\infty\notin X$ y se define $Y=X\cup\{\infty\}$
  con la topología generada por la subbase la cual consiste
  de $\mathcal{P}(X)$ y los conjuntos de la forma $C_i\cup\{\infty\}$.
  Sean $A = \{(x,x):x\in X\} \subseteq X\times Y$ y
  $K = \Adh(A)$. Por hipótesis, la proyección $\pi_2(K)$
  es cerrada en $Y$ y contiene a $X$. Entonces 
  $\infty \in \pi_2(K)$ ya que la propiedad de intersección finita de 
  la colección $(C_i)_{i\in I}$ garantiza que $\{\infty\}$ no sea un 
  conjunto abierto de $Y$ con lo que existe $x\in X$ tal que $(x,\infty)\in K$. 
  Esto es, para todo
  $i\in I$ y $U$ vecindad de $x$, $(U\times(C_i\cup\{\infty\}))\cap\Delta\neq\varnothing$,
  lo que implica que $U\cap C_i\neq\varnothing$ pues $x$ debe estar en ambos.
  Pero esto es $x\in\bigcap_{i\in I}C_i$.
\end{Demo}
