\begin{Teo}
  Si $X$ es un espacio compacto y $K\subseteq X$ es cerrado,
  entonces $K$ es compacto.
\end{Teo}
\begin{Demo}
  Sea $Y$ un espacio topológico. Por hipótesis la segunda proyección
  $\pi_2:X\times Y\longrightarrow Y$  es cerrada. Como $K$ es cerrado,
  entonces todo cerrado de $K$ es cerrado en $X$ y $\pi_2\mid_{K\times Y}$
  es también cerrada.
\end{Demo}
\begin{Teo}
  Sean $X$ un espacio de Hausdorff y $K\subseteq X$. Si $K$ es compacto,
  entonces es cerrado.
\end{Teo}
\begin{Demo}
  Como $X$ es Hausdorff, entonces $\Delta=\set{(x,x)}{x\in X}$ es cerrado en $X\times X$
  con lo que $\Delta_K=\Delta\cap(K\times X)$ es cerrado en $K\times X$. Por la compacidad
  de $K$, se tiene que en particular $\pi_2:K\times X\longrightarrow X$ es cerrada, luego
  $K=\pi_2(\Delta_K)=K$ es cerrado en $K$.
\end{Demo}
\begin{Teo}
  Sean $X,Y$ espacios topológicos y $f:X\longrightarrow Y$ una función continua.
  Si $X$ es compacto entonces $f(X)$ es compacto en $Y$.
\end{Teo}
\begin{Demo}
  Sea $Z$ un espacio topológico. Por hipótesis, la segunda proyección
  $p_2:X\times Z\longrightarrow Z$ es cerrada. Sean $\pi_2:f(X)\times Z\longrightarrow Z$
  la segunda proyección y $C\subseteq f(X)\times Z$ cerrado. El conjunto
  $A=\set{(x,z)\in X\times Z}{(f(x),z)\in C}$ es cerrado en $X\times Z$ pues,
  denotando por $I:Z\longrightarrow Z$ la función $z\longmapsto z$,
  $A$ es la imagen inversa de $C$ mediante la función $(x,z)\longmapsto(f(x),I(z))$ la cual
  es continua. Como $p_2(A)=\pi_2(C)$ se concluye que $p_2$ es cerrada y en consecuencia,
  $f(X)$ es compacto.
\end{Demo}
\begin{Teo}
  Sean $X_1,X_2$ espacios topológicos y $X=X_1\times X_2$. $X_1,X_2$
  son compactos si y solo si $X$ es compacto.
\end{Teo}
\begin{Demo}
\begin{enumerate}
  \item Supóngase que $X_1,X_2$ son compactos y sea $Y$ un espacio topológico.
  Por hipótesis, las segundas proyecciones $p_2:X_1\times(X_2\times Y)\longrightarrow(X_2\times Y)$
  y $q_2:X\times Y\longrightarrow Y$ son cerradas. Como $(X_1\times X_2)\times Y$
  es homeomorfo a $X_1\times(X_2\times Y)$ mediante la función
  $((x_1,x_2),y)\longmapsto(x_1,(x_2,y))$, entonces todo cerrado $C\subseteq(X_1\times X_2)\times Y$
  tiene un único cerrado asociado 
  $C^*=\set{(x_1,(x_2,y))}{((x_1,x_2),y)\in C}\subseteq X_1\times(X_2\times Y)$. 
  Por hipótesis, $p_2(C^*)$ es cerrado y enconsecuencia $q_2(p_2(C^*))$ es cerrado
  en $Z$. Nótese que la segunda proyección
  $\pi_2:(X_1\times X_2)\times Y\longrightarrow Y$ $q_2\circ p_2$ satisface que
  $\pi_2(C)=(q_2\circ p_2)(C^*)$. Así, $X$ es compacto.
\end{enumerate}
\end{Demo}
\begin{Teo}
  Sea $X$ un conjunto totalmente ordenado con la propiedad del supremo. Si $X$
  es dotado de la topología del orden, entonces todo intervalo cerrado de $X$
  es compacto.
\end{Teo}
\begin{Demo}
  Para evitar ambigüedad, las parejas $(x,y)\in X\times Y$ se denotarán por $x\times y$.
  La notación $(\cdot,\cdot)$ se reserva para intervalos abiertos de $X$.

  Sean $a,b\in X$ con $a\leq b$ y $Y$ un espacio topológico. Como $[a,b]$ es convexo,
  entonces la topología heredada coincide con la del orden. Para $y\in Y$ se denotará
  $\mathcal{V}(y)$ el conjunto de las vecindades de $y$.

  Sea $C\subseteq[a,b]\times Y$ cerrado y $y\in\Adh(\pi_2(C))$. Para $W\in\mathcal{V}(y)$,
  se define $A_W=\set{x\in[a,b]}{\Exists{w}[w\in W]{x\times w\in C}}$. Como
  $y\in\Adh(\pi_2(C))$, este conjunto es no vacío. En efecto, existe $p\in W\cap\pi_2(C)$,
  luego por sobreyectividad de la segunda proyección, existe $x\in [a,b]$ tal que
  $x\times p\in C$, con lo que $x\in A_W$. Evidentemente $A_W\subseteq[a,b]$,
  luego es acotado superiormente por $b$ y en consecuencia, $a_W=\sup A_W\in [a,b]$.
  Análogamente, el conjunto $\set{a_W}{W\in\mathcal{V}(y)}$ está contenido en $[a,b]$
  por lo anterior y $\alpha=\inf\set{a_W}{W\in\mathcal{V}(y)}\in[a,b]$.

  Sean $W\in\mathcal{V}(y)$ y $V$ una vecindad de $\alpha$ en $X$. Por lo anterior, $V\cap[a,b]$
  es una vecindad de $\alpha$ en $[a,b]$ y por definición de $\alpha$, existe
  una vecindad $W_1$ de $y$ tal que $\alpha\leq a_{W_1}\in V$ (¿Esto se puede si $X\neq \R$?).
  por propiedades del supremo, $a_{W_1\cap W}\leq a_{W_1}$
  (¿Hace falta que $W\cap W_1\neq\varnothing$?) con lo que
  $a_{W_1\cap W}\in V$ y existen $x\in V,w\in W_1\cap W\subseteq W$ tales que $x\times w\in C$,
  luego $\alpha\times y\in\Adh(C)=C$, es decir $y\in\pi_2(C)$.
\end{Demo}