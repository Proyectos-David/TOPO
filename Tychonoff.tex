\begin{Def}
  Una colección $\{(X_\alpha,f_{\alpha,\beta}):\Lambda\}$ es llamada
  \emph{sistema inverso} cuando cada $X_\alpha$ es un espacio topológico, $\Lambda$
  es dirigido por una relación $<$, para todo $\alpha<\beta$, las funciones
  $f_{\alpha,\beta}$ están definidas, son continuas, $f_{\beta,\alpha}\colon X_\alpha\longrightarrow X_\beta$
  y si $\alpha<\gamma<\beta$ entonces $f_{\gamma,\alpha}\circ f_{\beta,\gamma}=f_{\beta,\alpha}$.

  El límite inverso de este sistema es el subespacio de $\prod_{\alpha\in\Lambda}X_\alpha$ tal que
  para todo $(x_\alpha)_{\alpha\in\Lambda}$ en el límite y $\beta>\gamma$,
  \[f_{\beta,\gamma}((x_\alpha)_{\alpha\leq\beta})=(x_\alpha)_{\alpha\leq\gamma}\]
\end{Def}

\begin{Lema}
  Sean $\{X_\alpha\}_{\alpha\in J}$ una colección de espacios topológicos, $X=\prod_{\alpha\in J}X_\alpha$,
  $(x_\alpha)_{\alpha\in J}\in X$ y $A\subseteq X$. $(x_\alpha)_{\alpha\in J}\in\Adh(A)$ si y solo si
  para todo $I\subseteq J$ finito, 
  \[\pi_I((x_\alpha)_{\alpha\in J})=(x_\alpha)_{\alpha\in I}\in \Adh(\pi_I(A))\]
  Tomando este último como subconjunto de $\prod_{\alpha\in I}X_\alpha$ con la topología producto.
\end{Lema}
\begin{Demo}
  la primera implicación es trivial.
  Para la otra implicación, basta con demostrar que toda vecindad básica de $(x_\alpha)_{\alpha\in J}$
  interseca con $A$. Sea $U=\prod_{\alpha\in J}U_\alpha$ un abierto básico al cual pertenece
  $(x_\alpha)_{\alpha\in J}$ y $I\subseteq J$ el conjunto de índices para los que
  $\alpha\in I$ implica que $U_\alpha\neq X_\alpha$. Como $I$ es finito, por hipótesis,
  $\pi_I((x_\alpha)_{\alpha\in J})\in\Adh(\pi_I(A))$ luego
  \[\prod_{\alpha\in I} U_\alpha\cap \pi_I(A)\neq\varnothing\]
  Como $\alpha\not\in I$ implica que $U_\alpha=X_\alpha$, entonces trivialmente
  \[U\cap A\neq \varnothing\]
\end{Demo}

\begin{Teo}
  Sea $\{X_\alpha\}_{\alpha<\kappa}$ una colección de compactos indexados por un ordinal
  $\kappa$. El espacio $X=\prod_{\alpha<\kappa}X_\alpha$ es compacto bajo la topología
  producto.
\end{Teo}
\begin{Demo}
  La demostración se realizará por inducción transfinita. El caso base,
  $\kappa=0$ es trivial. Antes de continuar con el paso inductivo,
  se introduce una notación que será de utilidad:
  
  Para $\gamma\leq\kappa$ se denotará
  $X^\gamma=\left(\prod_{\alpha<\gamma}X_\alpha\right)\times Y$. Se toma la convención de que
  $X^0=Y$.
  Para $\beta<\gamma$ se define
  \[\pi_\beta^\gamma:X^\gamma\longrightarrow X^\beta\]
  por $(y,(x_\alpha)_{\alpha<\gamma})\longmapsto(y,(x_\alpha)_{\alpha<\beta})$. Sea
  $K\subseteq X^\kappa$ cerrado. Se denotará $K_\beta=\Adh\left(\pi_\beta^\kappa(K)\right)$,
  con lo que $K_\kappa=K$ y $K_0=\Adh\left(\pi_0^\beta(K)\right)$ sería la segunda
  proyección en $X\times Y$. Basta con demostrar justamente que $K_0=\pi_0^\beta(K)$.

  Para la hipótesis de inducción, se va a suponer que, para todo $\beta<\kappa$,
  la segunda proyección
  \[\pi_2\colon \left(\prod_{\alpha<\beta}X_\alpha\right)\times Y\longrightarrow Y\]
  es cerrada. Esto es equivalente a que la función $\pi_0^\beta$ sea cerrada.
  Es decir, dado $x_0\in K_0$, existe $x_\beta\in K_\beta$ tal que
  $\pi_0^\beta(x_\beta)=x_0$.

  Si $\kappa$ es un sucesor ($\kappa=\beta+1)$, entonces por hipótesis, la proyección
  \[\pi_2\colon X_\beta\times X^\beta\longrightarrow X^\beta\]
  es cerrada. Esto es equivalente a que la función $\pi_\beta^\kappa$ es cerrada.
 
  \begin{longderivation}<.8>
      & K_\beta=\pi_\beta^\kappa(K)\\
    \To\\
      & \pi_0^\beta(K_\beta)=(\pi_0^\beta\circ\pi_\beta^\kappa)(K)\\
    \equiv\\
      & K_0 = \pi_0^\kappa(K)
  \end{longderivation}

  Si $\kappa$ es un ordinal límite (no es un sucesor), se considera
  el sistema inverso de espacios $\set{(X^\gamma,\pi_\beta^\gamma)}{\kappa}$.
  Se comprueba fácilmente que este es un sistema inverso pues las
  funciones dadas son efectivamente continuas y para $\gamma>\beta>\mu$,
  \[\pi_\mu^\beta\circ\pi_\beta^\gamma=\pi_\mu^\gamma\]
  Con esto, el espacio $X^\kappa$ es justamente el límite inverso
  de este sistema. La razón está justamente en el comportamiendo
  de las funciones dadas, lo cual fue descrito al momento de introducirlas.

  Debido a esto, para todo $\beta<\kappa$ cualquier elemento de la forma
  $(x_\alpha)_{\alpha<\beta}$ define un elemento $x_\kappa\in X^\kappa$.

  Por hipótesis, basta con demostrar que los $x_\kappa$ definidos
  por elementos $x_\beta\in K_\beta$, están en $K$. Por el lema anterior,
  basta con considerar $F$ una colección finita de ordinales menores a $\kappa$.
  \begin{longderivation}<.8>
      & \pi_F(x_\kappa)\\
    =\\
      & (\pi_F^\beta\circ\pi_\beta^\kappa)(x_\kappa)\\
    =\\
      & \pi_F^\beta(x_\beta)\\
    \in\\
      & \pi_F^\beta(K_\beta)\\
    =\\
      & \pi_F(\Adh(\pi_\beta^\kappa(K)))\\
    \subseteq\\
      & \Adh((\pi_F\circ\pi_\beta^\kappa)(K))\\
    =\\
      & \Adh(\pi_F(K))
  \end{longderivation}
\end{Demo}