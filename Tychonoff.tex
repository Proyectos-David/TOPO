\begin{Lema}
  Sean $\{X_\alpha\}_{\alpha\in J}$ una colección de espacios topológicos, $X=\prod_{\alpha\in J}X_\alpha$,
  $(x_\alpha)_{\alpha\in J}\in X$ y $A\subseteq X$. $(x_\alpha)_{\alpha\in J}\in\Adh(A)$ si y solo si
  para todo $I\subseteq J$ finito, 
  \[\pi_I((x_\alpha)_{\alpha\in J})=(x_\alpha)_{\alpha\in I}\in \Adh(\pi_I(A))\]
  Tomando este último como subconjunto de $\prod_{\alpha\in I}X_\alpha$ con la topología producto.
\end{Lema}
\begin{Demo}
  la primera implicación es trivial.
  Para la otra implicación, basta con demostrar que toda vecindad básica de $(x_\alpha)_{\alpha\in J}$
  interseca con $A$. Sea $U=\prod_{\alpha\in J}U_\alpha$ un abierto básico al cual pertenece
  $(x_\alpha)_{\alpha\in J}$ y $I\subseteq J$ el conjunto de índices para los que
  $\alpha\in I$ implica que $U_\alpha\neq X_\alpha$. Como $I$ es finito, por hipótesis,
  $\pi_I((x_\alpha)_{\alpha\in J})\in\Adh(\pi_I(A))$ luego
  \[\prod_{\alpha\in I} U_\alpha\cap \pi_I(A)\neq\varnothing\]
  Como $\alpha\not\in I$ implica que $U_\alpha=X_\alpha$, entonces trivialmente
  \[U\cap A\neq \varnothing\]
\end{Demo}

\begin{Teo}
  Sea $\{X_\alpha\}_{\alpha<\kappa}$ una colección de compactos indexados por un ordinal
  $\kappa$. El espacio $X=\prod_{\alpha<\kappa}X_\alpha$ es compacto bajo la topología
  producto.
\end{Teo}
\begin{Demo}
  La demostración se realizará por inducción transfinita. Para $\gamma\leq\kappa$ se denotará
  $X^\gamma=\left(\prod_{\alpha<\gamma}X_\alpha\right)\times Y$ con lo que $X^0=Y$. Para $\beta<\gamma$ se define
  \[\pi_\beta^\gamma:X^\gamma\longrightarrow X^\beta\]
  por $(y,(x_\alpha)_{\alpha<\gamma})\longmapsto(y,(x_\alpha)_{\alpha<\beta})$. Sea
  $K\subseteq X^\kappa$ cerrado. Se denotará $K_\beta=\Adh\left(\pi_\beta^\kappa(K)\right)$,
  con lo que $K_\kappa=K$ y $K_0=\Adh\left(\pi_0^\beta(K)\right)$ sería la segunda
  proyección en $X\times Y$. Basta con demostrar justamente que $K_0=\pi_0^\beta(K)$.

  Se demuestra que, dado $x_0\in K_0$, para todo $\beta<\kappa$, existe $x_\beta\in X^\gamma$ tal que
  para todo $\gamma$ con $\beta<\gamma<\kappa$, $\pi_\beta^\gamma(x_\gamma)=x_\beta$.(¿Qué es $x_\gamma$ específicamente?)
  En particular, $\pi_0^\beta(x_\beta)=x_0$.

  El caso base es trivial ($\beta=0$). Si $\kappa$ es sucesor de algún $\beta$, entonces
  \[\pi_\beta^\kappa:X_\beta\times X^\beta\longrightarrow X^\beta\]
  (¿por qué la misma notación?) es cerrada ya que $X^\beta$ es compacto, luego $\pi_\beta^\kappa(K)=K_\beta$
  y como $K$ es cerrado, existe $x_\kappa\in K$ con $\pi_\beta^\kappa(x_\kappa)=x_\beta$ con lo que
  \[\pi_0^\kappa(x_\kappa) = \left(\pi_0^\beta\circ\pi_\beta^\kappa\right)(x_\kappa)=x_0\]

  Si $\kappa$ es un ordinal límite (no es sucesor), entonces se puede considerar $X^\kappa$ como
  el límite de los espacios $X^\beta$ con las proyecciones $\pi_\beta^\gamma$ entre si.
  Debido a esto, todo elemento $(x_\beta)_{\beta<\kappa}$ define un elemento $x_\kappa\in X^\kappa$.
  Basta con demostrar que $x_\kappa\in K$. Por el lema anterior, basta con considerar
  cualquier colección finita $F$ de ordinales menores a $\kappa$. Existe un ordinal
  $\beta$ el cual es mayor a todos los ordinales en $F$, con lo que
  \begin{longderivation}
      & \pi_F(x_\kappa)\\
    =\\
      & (\pi_F\beta\circ\pi_\beta^\kappa)(x_\kappa)\\
    =\\
      & \pi_F^\beta(x_\beta)\\
    \in\\
      & \pi_F^\beta(K_\beta)\\
    =\\
      & \pi_F^\beta(\Adh(\pi_\beta^\kappa(K)))\\
    \subseteq\\
      & \Adh((\pi_F^\beta\circ\pi_\beta^\kappa(K)))\\
    =\\
      & \Adh(\pi_F(K))
  \end{longderivation}
\end{Demo}