\documentclass{article}
\usepackage{analysis} % Sintaxis del análisis
\usepackage{logicDG}  % Para lógica calculatoria
\usepackage{upgreek}  % Mejores letras griegas
\usepackage{enumitem} % Listas
\usepackage{amsfonts} % Letras Caligráficas
\usepackage{mathrsfs} % Teoremas
\usepackage{amsthm}   % Teoremas
\usepackage{thmtools} % Teoremas
\usepackage[hidelinks]{hyperref}  % Links a referencias
\usepackage[T1]{fontenc}  % Fuente
\usepackage{lmodern}  % Fuente
\usepackage[spanish,es-noquoting,es-lcroman,es-noshorthands]{babel}
\usepackage[explicit,pagestyles]{titlesec}
\usepackage{setspace}
\usepackage{geometry}
\usepackage{csquotes}
\usepackage[sorting=none,
style=phys,
maxnames=999,
biblabel=brackets,
pageranges=false,
chaptertitle=false,
giveninits=false]{biblatex}

%%% Página
\geometry{
  left=2cm,
  right=2cm,
  bottom=4cm,
  a4paper
}
\newpagestyle{main}[]
{
  \sethead[\thepage][][\itshape\sectiontitle]
  {\itshape\sectiontitle}{}{\thepage}
  \headrule
}
\pagestyle{main}
\titleformat{\section}[block]
  {\Large\itshape\bfseries}
  {}
  {0pt}
  {
    \thesection\hspace{15pt}#1
  }
\titleformat{name=\section,numberless}[display]
  {\Large\itshape\bfseries}
  {}
  {0pt}
  {
      #1
  }

\titleformat{\subsection}[block]
  {\large\itshape\bfseries}
  {}
  {0pt}
  {
    \thesubsection\hspace{10pt}#1
  }
\titleformat{name=\subsection,numberless}[display]
  {\large\itshape\bfseries}
  {}
  {0pt}
  {
      #1
  }
\everymath{\displaystyle}
\RenewDocumentCommand{\dfrac}{mm}{\frac{\displaystyle#1}{\displaystyle#2}}
\doublespacing
\hyphenpenalty=10000
\relpenalty=10000
\binoppenalty=10000
\interfootnotelinepenalty=10000


%%%% Teoremas, Definiciones y demás
\declaretheoremstyle[
  spaceabove=10pt,
  spacebelow=20pt,
  bodyfont={\normalfont},
  notefont={\normalfont},
  notebraces={(}{)},
  headpunct={:},
  headfont={\bfseries},
  ]{definicion}
  
\declaretheoremstyle[
    headpunct={:},
    headfont={\bfseries},
    bodyfont={\normalfont},
    qed={\qedsymbol}
]{Proof}

\declaretheorem[style=definicion,name=Definición]{Def}
\declaretheorem[style=definicion,name=Teorema,numberlike=Def]{Teo}
\declaretheorem[style=definicion,name=Corolario,numberlike=Def]{Coro}
\declaretheorem[style=Proof, numbered=no, name=Demostración]{Demo}
\declaretheorem[style=definicion, numberlike=Def, name=Lema]{Lema}
\declaretheorem[style=definicion,name=Ejemplo,numberlike=Def]{Ej}
\declaretheorem[style=definicion,name=Nota,numberlike=Def]{nota}

%% Listas
\setlist[enumerate]{label=(\roman*),ref=(\roman*),wide=0pt,listparindent=\parindent}
\setlist[itemize]{label=\textbullet,wide=0pt}

%%% Cambios de letras y simbolos
\renewcommand{\epsilon}{\upvarepsilon}

%%% Autor y Fecha
\author{David Gómez, Laura Rincón}
\date{\today}
\title{La Caracterización de Kuratowski-Mrówa Para Conjuntos Compactos}
\begin{document}
\maketitle
Por ahora cualquier cosa\dots
\begin{Def}[Kuratowski-Mrówka]
  Un espacio topológico $X$ es compacto si y solo si
  para todo espacio topológico $Y$, la segunda proyección
  $\pi_2:X\times Y\longrightarrow Y$ es cerrada.
\end{Def}

Antes de demostrar la equivalencia con la definición usual, se presenta
un lema el cual facilita entender este enunciado y la demostración.

\begin{Lema}
  Sean $X,Y$, espacios topológicos y $f:X\longrightarrow Y$.
  $f$ es cerrada si y solo si, para todo $y\in Y$, $U\subseteq X$
  tal que $f^{-1}(\{y\})\subseteq U$, existe una vecindad $V$ de $y$
  tal que $f^{-1}(V)\subseteq U$.
\end{Lema}
\begin{Demo}
  \begin{enumerate}
    \item Supóngase que $f$ es cerrada y sean $y\in Y$, $U\subseteq X$ tal que $f^{-1}(\{y\})\subseteq U$.
    Entones $X-U$ es cerrado y $f(X-U)$ también. Se define $V=Y-f(X-U)$ que es abierto en $Y$ y además
    $y\in V$, pues de no ser el caso, $y\in f(X-U)\cap f(U)=\varnothing$.

    Se afirma que $f^{-1}(V)\subseteq U$: sea $p\in f^{-1}(V)$, entonces $f(p)\not\in f(X-U)$ y
    $p\not\in X-U$, es decir $p\in U$.
    \item Supóngase que $f$ satisface la hipótesis del recíproco y sean $C\subseteq X$ cerrado, $y\not\in f(C)$.
    Como $C$ es cerrado, entonces $X-U$ es un abierto el cual contiene a $f^{-1}(\{y\})$ y por hipótesis,
    se cuenta con una vecindad $V$ de $y$ tal que $f^{-1}(V)\subseteq X-C$, con lo que
    $V\cap f(C)=\varnothing$ y $y\in\Int(Y-f(C))$. Así, $f(C)$ es cerrado.
  \end{enumerate}
\end{Demo}

\begin{Lema}[Lema del Tubo]
  Sean $X,Y$ espacios topológicos, $y\in Y$ y $W\subseteq X\times Y$ abierto tal que
  $X\times\{y\}\subseteq W$. Si $X$ es compacto entonces existe una vecindad $V$ de $y$ tal
  que $X\times V\subseteq W$.
\end{Lema}
\begin{Demo}
  Dado que $X\times\{y\}$ es homeomorfo a $X$, el cual es compacto, se cuenta con
  una cantidad finita de abiertos básicos $U_1\times E_1,\dots,U_n\times E_n\subseteq W$
  los cuales cubren a $X\times\{y\}$. Sin pérdida de generalidad supóngase que
  cada uno de estos abiertos interseca al conjunto. Se define
  $V=\bigcap_{i=1}^n E_i$ que es abierto en $Y$ y $y\in V$. Sea $(x,z)\in X\times V$.
  Entonces para algún $i\in\{1,\dots,n\}$, $(x,y)\in U_i\times E_i$ y como
  $z\in\bigcap_{i=1}^n E_i$, entonces $(x,z)\in U_i\times E_i$. Así,
  $X\times W\subseteq\bigcup_{i=1}^nU_i\times E_i\subseteq W$.
\end{Demo}

Aplicando el primer lema, el enunciado el teorema presentado resulta equivalente
al siguiente

\begin{Teo}[Kuratowski-Mrówka modificado]
  Un espacio topológico $X$ es compacto (en el sentido de las cubiertas abiertas) si y solo si
  para todo espacio topológico $Y$, se satisface lo siguiente: para todo $y\in Y$,
  $W\subseteq X\times Y$ abierto tal que $X\times\{y\}\subseteq W$, existe una vecindad
  $V$ de $y$ la cual cumple que $X\times V\subseteq U$.
\end{Teo}
\begin{Demo}
  La primera implicación es el lema del tubo.

  Supóngase que X que satisface la propiedad y
  sea $(C_i)_{i\in I}$ una colección de conjuntos 
  cerrados de $X$ la cual satisface la propiedad de las intersecciones 
  finitas (esto es que cualquier intersección finita 
  de los $C_i$ es no vacía). Para demostrar que $X$ es 
  compacto, basta con demostrar que $\bigcap_{i\in I}C_i\neq \varnothing$
  
  Se considera $\infty\notin X$ y se define $Y=X\cup\{\infty\}$
  con la topología generada por la subbase la cual consiste
  de $\mathcal{P}(X)$ y los conjuntos de la forma $C_i\cup\{\infty\}$.
  Sean $A = \{(x,x):x\in X\} \subseteq X\times Y$ y
  $K = \Adh(A)$. Por hipótesis, la proyección $\pi_2(K)$
  es cerrada en $Y$ y contiene a $X$. Entonces 
  $\infty \in \pi_2(K)$ ya que la propiedad de intersección finita de 
  la colección $(C_i)_{i\in I}$ garantiza que $\{\infty\}$ no sea un 
  conjunto abierto de $Y$ con lo que existe $x\in X$ tal que $(x,\infty)\in K$. 
  Esto es, para todo
  $i\in I$ y $U$ vecindad de $x$, $(U\times(C_i\cup\{\infty\}))\cap\Delta\neq\varnothing$,
  lo que implica que $U\cap C_i\neq\varnothing$ pues $x$ debe estar en ambos.
  Pero esto es $x\in\bigcap_{i\in I}C_i$.
\end{Demo}

\begin{Teo}
  Si $X$ es un espacio compacto y $K\subseteq X$ es cerrado,
  entonces $K$ es compacto.
\end{Teo}
\begin{Demo}
  Sea $Y$ un espacio topológico. Por hipótesis la segunda proyección
  $\pi_2:X\times Y\longrightarrow Y$  es cerrada. Como $K$ es cerrado,
  entonces todo cerrado de $K$ es cerrado en $X$ y $\pi_2\mid_{K\times Y}$
  es también cerrada.
\end{Demo}
\begin{Teo}
  Sean $X$ un espacio de Hausdorff y $K\subseteq X$. Si $K$ es compacto,
  entonces es cerrado.
\end{Teo}
\begin{Demo}
  Como $X$ es Hausdorff, entonces $\Delta=\set{(x,x)}{x\in X}$ es cerrado en $X\times X$
  con lo que $\Delta_K=\Delta\cap(K\times X)$ es cerrado en $K\times X$. Por la compacidad
  de $K$, se tiene que en particular $\pi_2:K\times X\longrightarrow X$ es cerrada, luego
  $K=\pi_2(\Delta_K)=K$ es cerrado en $K$.
\end{Demo}
\begin{Teo}
  Sean $X,Y$ espacios topológicos y $f:X\longrightarrow Y$ una función continua.
  Si $X$ es compacto entonces $f(X)$ es compacto en $Y$.
\end{Teo}
\begin{Demo}
  Sea $Z$ un espacio topológico. Por hipótesis, la segunda proyección
  $p_2:X\times Z\longrightarrow Z$ es cerrada. Sean $\pi_2:f(X)\times Z\longrightarrow Z$
  la segunda proyección y $C\subseteq f(X)\times Z$ cerrado. El conjunto
  $A=\set{(x,z)\in X\times Z}{(f(x),z)\in C}$ es cerrado en $X\times Z$ pues,
  denotando por $I:Z\longrightarrow Z$ la función $z\longmapsto z$,
  $A$ es la imagen inversa de $C$ mediante la función $(x,z)\longmapsto(f(x),I(z))$ la cual
  es continua. Como $p_2(A)=\pi_2(C)$ se concluye que $p_2$ es cerrada y en consecuencia,
  $f(X)$ es compacto.
\end{Demo}
\begin{Teo}
  Sean $X_1,X_2$ espacios topológicos y $X=X_1\times X_2$. $X_1,X_2$
  son compactos si y solo si $X$ es compacto.
\end{Teo}
\begin{Demo}
\begin{enumerate}
  \item Supóngase que $X_1,X_2$ son compactos y sea $Y$ un espacio topológico.
  Por hipótesis, las segundas proyecciones $p_2:X_1\times(X_2\times Y)\longrightarrow(X_2\times Y)$
  y $q_2:X\times Y\longrightarrow Y$ son cerradas. Como $(X_1\times X_2)\times Y$
  es homeomorfo a $X_1\times(X_2\times Y)$ mediante la función
  $((x_1,x_2),y)\longmapsto(x_1,(x_2,y))$, entonces todo cerrado $C\subseteq(X_1\times X_2)\times Y$
  tiene un único cerrado asociado 
  $C^*=\set{(x_1,(x_2,y))}{((x_1,x_2),y)\in C}\subseteq X_1\times(X_2\times Y)$. 
  Por hipótesis, $p_2(C^*)$ es cerrado y enconsecuencia $q_2(p_2(C^*))$ es cerrado
  en $Z$. Nótese que la segunda proyección
  $\pi_2:(X_1\times X_2)\times Y\longrightarrow Y$ $q_2\circ p_2$ satisface que
  $\pi_2(C)=(q_2\circ p_2)(C^*)$. Así, $X$ es compacto.
\end{enumerate}
\end{Demo}
\begin{Teo}
  Sea $X$ un conjunto totalmente ordenado con la propiedad del supremo. Si $X$
  es dotado de la topología del orden, entonces todo intervalo cerrado de $X$
  es compacto.
\end{Teo}
\begin{Demo}
  Para evitar ambigüedad, las parejas $(x,y)\in X\times Y$ se denotarán por $x\times y$.
  La notación $(\cdot,\cdot)$ se reserva para intervalos abiertos de $X$.

  Sean $a,b\in X$ con $a\leq b$ y $Y$ un espacio topológico. Como $[a,b]$ es convexo,
  entonces la topología heredada coincide con la del orden. Para $y\in Y$ se denotará
  $\mathcal{V}(y)$ el conjunto de las vecindades de $y$.

  Sea $C\subseteq[a,b]\times Y$ cerrado y $y\in\Adh(\pi_2(C))$. Para $W\in\mathcal{V}(y)$,
  se define $A_W=\set{x\in[a,b]}{\Exists{w}[w\in W]{x\times w\in C}}$. Como
  $y\in\Adh(\pi_2(C))$, este conjunto es no vacío. En efecto, existe $p\in W\cap\pi_2(C)$,
  luego por sobreyectividad de la segunda proyección, existe $x\in [a,b]$ tal que
  $x\times p\in C$, con lo que $x\in A_W$. Evidentemente $A_W\subseteq[a,b]$,
  luego es acotado superiormente por $b$ y en consecuencia, $a_W=\sup A_W\in [a,b]$.
  Análogamente, el conjunto $\set{a_W}{W\in\mathcal{V}(y)}$ está contenido en $[a,b]$
  por lo anterior y $\alpha=\inf\set{a_W}{W\in\mathcal{V}(y)}\in[a,b]$.

  Sean $W\in\mathcal{V}(y)$ y $V$ una vecindad de $\alpha$ en $X$. Por lo anterior, $V\cap[a,b]$
  es una vecindad de $\alpha$ en $[a,b]$ y por definición de $\alpha$, existe
  una vecindad $W_1$ de $y$ tal que $\alpha\leq a_{W_1}\in V$ (¿Esto se puede si $X\neq \R$?).
  por propiedades del supremo, $a_{W_1\cap W}\leq a_{W_1}$
  (¿Hace falta que $W\cap W_1\neq\varnothing$?) con lo que
  $a_{W_1\cap W}\in V$ y existen $x\in V,w\in W_1\cap W\subseteq W$ tales que $x\times w\in C$,
  luego $\alpha\times y\in\Adh(C)=C$, es decir $y\in\pi_2(C)$.
\end{Demo}
\end{document}
