\documentclass[10pt]{beamer}

\usepackage[arrows]{logicDG}
\usepackage{analysis}
\usepackage{nicefrac}
\usepackage{amsfonts}
\usepackage{upgreek}
\usepackage{mathrsfs}
\usepackage{thmtools}
\usepackage{marvosym}
\usepackage{letltxmacro}
\usepackage{biblatex}
\addbibresource{Referencias.bib}
\usepackage{csquotes}
\usepackage[spanish, es-noquoting, es-lcroman, es-noshorthands]{babel}
\usetheme{Warsaw}
\setbeamercovered{transparent}

\makeatletter
\let\oldr@@t\r@@t
\def\r@@t#1#2{%
\setbox0=\hbox{$\oldr@@t#1{#2\,}$}\dimen0=\ht0
\advance\dimen0-0.2\ht0
\setbox2=\hbox{\vrule height\ht0 depth -\dimen0}%
{\box0\lower0.4pt\box2}}
\LetLtxMacro{\oldsqrt}{\sqrt}
\renewcommand*{\sqrt}[2][\ ]{\oldsqrt[#1]{#2}}
\makeatother
\renewcommand{\epsilon}{\upvarepsilon}

\title{La Caracterización de Kuratowski-Mrówa Para Conjuntos Compactos}
\author[David G., Laura R.]{David Gómez, Laura Rincón}
\date[26/05/2025]{26 de Mayo de 2025}

\begin{document}
  \begin{frame}
    \titlepage
  \end{frame}

  \section{La Caracterización de Kuratowski-Mrówa}
  \begin{frame}{La Caracterización de Kuratowski-Mrówa}
      \begin{definition}
      Un espacio topológico $X$ es compacto si y solo si para todo espacio
      espacio topológico $Y$, la segunda proyección $\pi_2: X\times Y \longrightarrow Y$
      es cerrada 
    \end{definition}
  \end{frame}
  \begin{frame}{Definición usual}
    \begin{theorem}
      Un espacio $X$ es compacto si toda cubierta abierta de $X$
      tiene una subcubierta finita
    \end{theorem}
  \end{frame}

  \begin{frame}{Lemas preliminares}
    \begin{lemma}
        Sean $X,Y$, espacios topológicos y $f:X\longrightarrow Y$.
        $f$ es cerrada si y solo si, para todo $y\in Y$ y todo $U\subseteq X$ abierto
        tal que $f^{-1}(\{y\})\subseteq U$, existe una vecindad $V$ de $y$
        tal que $f^{-1}(V)\subseteq U$.
      \end{lemma}
      \begin{definition}
      Un espacio topológico $X$ es compacto si y solo si para todo espacio
      espacio topológico $Y$, la segunda proyección $\pi_2: X\times Y \longrightarrow Y$
      es cerrada 
    \end{definition}
    \begin{theorem}[Kuratowski-Mrówa modificado]
  Un espacio topológico $X$ es compacto (por la definición usual) si y solo si
  dado un espacio $Y$, se satisface lo siguiente: para todo $y\in Y$ y
  $U\subseteq X\times Y$ abierto tal que $X\times \{y\}\subseteq U$, existe una vecindad
  $V$ de $y$ la cual cumple que $X \times V\subseteq U$.
\end{theorem}
\end{frame}
\begin{frame}{Demostración} 
\only<1>{
  \begin{lemma}[Lema del tubo]
    Sean $X,Y$ espacios topológicos, $y\in Y$ y $W\subseteq X\times Y$ abierto tal que
    $X\times\{y\}\subseteq W$. Si $X$ es compacto entonces existe una vecindad $V$ de $y$ tal
    que $X\times V\subseteq W$.
  \end{lemma}
}
\only<2>{
Sea $\{C_i\}_{i\in I}$ una colección de cerrados de $X$ con la propiedad
de las intersecciones finitas. Basta con demostrar que $\bigcap_{i\in I}C_i\neq\varnothing$.
Se considera $Y=X\cup\{\infty\}$ con $\infty\notin X$. $Y$ es dotado
con la topología generada por la subbase la cual consiste de $\mathcal{P}(X)$
y todos los conjuntos de la forma $C_i\cup\{\infty\}$.
En $X\times Y$, se consideran $\Delta=\set{(x,x)}{x\in X}$ y $K=\Adh(\Delta)$. Por
hipótesis, $\pi_2(K)$ es cerrado en $Y$ y $X\subseteq\pi_2(K)$.
$\infty\in \pi_2(K)$, debido a la propiedad de intersecciones finitas, ya que esta
implica que $\{\infty\}$ no es abierto en $Y$ y toda vecindad $(C_i\cup \{\infty\})\times U$
de $(x,\infty)$ interseca con $\Delta$. Esto es, para todo
$i\in I$ y $U$ vecindad de $x$, $(U\times(C_i\cup\{\infty\}))\cap\Delta\neq\varnothing$,
lo que implica que $U\cap C_i\neq\varnothing$ pues $x$ debe estar en ambos.
Pero esto es $x\in\bigcap_{i\in I}C_i$.
}
\end{frame}
\section{Algunos teoremas}
\begin{frame}{Algunos teoremas}
  
    \begin{theorem}
  Si $X$ es un espacio compacto y $K\subseteq X$ es cerrado,
  entonces $K$ es compacto.
\end{theorem}
Sea $Y$ un espacio topológico. Por hipótesis la segunda proyección
  $\pi_2:X\times Y\longrightarrow Y$  es cerrada. Como $K$ es cerrado,
  entonces todo cerrado de $K$ es cerrado en $X$ y $\pi_2\mid_{K\times Y}$
  es también cerrada.
\end{frame}
\begin{frame}
  \begin{theorem}
  Sean $X$ un espacio de Hausdorff y $K\subseteq X$. Si $K$ es compacto,
  entonces es cerrado.
\end{theorem}
Como $X$ es Hausdorff, entonces $\Delta=\set{(x,x)}{x\in X}$ es cerrado en $X\times X$
  con lo que $\Delta_K=\Delta\cap(K\times X)$ es cerrado en $K\times X$. Por la compacidad
  de $K$, se tiene que en particular $\pi_2:K\times X\longrightarrow X$ es cerrada, luego
  $K=\pi_2(\Delta_K)$ es cerrado en $X$.
\end{frame}
\begin{frame}
  \begin{theorem}
  Sean $X,Y$ espacios topológicos y $f:X\longrightarrow Y$ una función continua.
  Si $X$ es compacto entonces $f(X)$ es compacto en $Y$.
\end{theorem}


  Sea $Z$ un espacio topológico. Por hipótesis, la segunda proyección
  $p_2:X\times Z\longrightarrow Z$ es cerrada. Sean $\pi_2:f(X)\times Z\longrightarrow Z$
  la segunda proyección y $C\subseteq f(X)\times Z$ cerrado. El conjunto
  $A=\set{(x,z)\in X\times Z}{(f(x),z)\in C}$ es cerrado en $X\times Z$ pues,
  denotando por $I:Z\longrightarrow Z$ la función $z\longmapsto z$,
  $A$ es la imagen inversa de $C$ mediante la función $(x,z)\longmapsto(f(x),I(z))$ la cual
  es continua. Como $p_2(A)=\pi_2(C)$ se concluye que $p_2$ es cerrada y en consecuencia,
  $f(X)$ es compacto.
\end{frame}
\begin{frame}
  \begin{theorem}
  Sea $X$ un conjunto totalmente ordenado con la propiedad del supremo. Si $X$
  es dotado con la topología del orden, entonces todo intervalo cerrado de $X$
  es compacto.
\end{theorem}
  Denotando $\mathcal{V}(y)$ el conjunto de vecindades de $y$.
  Sea $C\subseteq[a,b]\times Y$ cerrado y $y\in\Adh(\pi_2(C))$. Para $W\in\mathcal{V}(y)$,
  se define $A_W=\set{x\in[a,b]}{\Exists{w}[w\in W]{x\times w\in C}}$,
  $a_W=\sup A_W$ y $\alpha=\inf\set{a_W}{W\in\mathcal{V}(y)}$
  Sean $V\cap [a,b]$ una vecindad de $\alpha$ en $[a,b]$ con $\alpha \neq \sup(V)$
  y $W$ una vecindad de $y$. Como $\alpha \neq \sup(V)$ entonces existe $W_1 \in \mathcal{V}(y)$
  tal que $\alpha \leq a_{W_1} < sup(V)$.
  
  Si $\alpha < a_{W_1\cap W}$ entonces $\alpha$ no es una cota superior de $A_{W_1\cap W}$,
  luego existe $x\in A_{W_1\cap W}$ tal que $\alpha < x \leq a_{W_1\cap W}$. Es claro que 
  $x \in V$ y que existe $w \in W_1\cap W \subseteq W$ tal que $x\times w \in K$. 
  Además $x\times w \in (V\times W)\cap K$. 
\end{frame}
  \section{Teorema de Tychonoff}
\begin{frame}{Tychonoff Caso finito}
  \onslide<1->{
  \begin{theorem}
  Sean $X_1,X_2$ espacios topológicos y $X=X_1\times X_2$. $X_1,X_2$
  son compactos si y solo si $X$ es compacto.
\end{theorem}
Supóngase que $X_1,X_2$ son compactos y sea $Y$ un espacio topológico.
  Por hipótesis, las segundas proyecciones $p_2:X_1\times(X_2\times Y)\longrightarrow(X_2\times Y)$
  y $q_2:X\times Y\longrightarrow Y$ son cerradas. Como $(X_1\times X_2)\times Y$
  es homeomorfo a $X_1\times(X_2\times Y)$ mediante la función
  $((x_1,x_2),y)\longmapsto(x_1,(x_2,y))$, entonces todo cerrado $C\subseteq(X_1\times X_2)\times Y$
  tiene un único cerrado asociado 
  $C^*=\set{(x_1,(x_2,y))}{((x_1,x_2),y)\in C}\subseteq X_1\times(X_2\times Y)$. 
  Por hipótesis, $p_2(C^*)$ es cerrado y enconsecuencia $q_2(p_2(C^*))$ es cerrado
  en $Z$. Nótese que la segunda proyección
  $\pi_2:(X_1\times X_2)\times Y\longrightarrow Y$ $q_2\circ p_2$ satisface que
  $\pi_2(C)=(q_2\circ p_2)(C^*)$. Así, $X$ es compacto.
  }
\end{frame}
\begin{frame}{Tychonoff Caso general}
\only<1>{\begin{theorem}
  Sea $\{X_\alpha\}_{\alpha<\kappa}$ una colección de espacios compactos
  ordenados por un ordinal $\kappa$. El espacio $\prod_{\alpha<\kappa}X_\alpha$
  es compacto bajo la topología producto.
 \end{theorem}} 
 \only<2>{
  Sea $Y$ un espacio topológico.
  Para $\gamma<\kappa$ se define $X^\gamma=\left(\prod_{\alpha<\gamma}X_\alpha\right)\times Y$
  y se toma la convención de que $X^0=Y$. Para $\beta<\gamma<\kappa$,
  se considera $\pi_\beta^\gamma:X^\gamma\longrightarrow X^\beta$ definida por
  \[\pi_\beta^\gamma((x_\alpha)_{\alpha<\gamma},y)=((x_\alpha)_{\alpha<\beta},y)\]

  Sea $K\subseteq X^\kappa$ cerrado. Para $\beta<\kappa$ se denota
  $K_\beta=\Adh(\pi_\beta^\gamma(K))$. $K=K_\kappa$ y basta con demostrar que
  $K_0=\pi_0^\kappa(K)$
 }
\end{frame}
\end{document}
